\documentclass[12pt,letterpaper]{article}
\usepackage[utf8]{inputenc}
\usepackage{natbib}
\usepackage{graphicx}
\usepackage{indentfirst}
\usepackage{indentfirst}
\usepackage[left=3cm,right=2cm,top=2cm,bottom=3cm]{geometry}

\usepackage{lipsum}

\setlength{\parindent}{2cm}
\setlength{\parskip}{\baselineskip}
\renewcommand{\baselinestretch}{1.5}

\renewcommand\contentsname{Índice de contenidos}
\renewcommand\refname{Referencias}

\begin{document}

\newpage
\vspace*{-.5cm}
\begin{picture}(18,4)(0,40)
	\put(350,-20){\includegraphics[scale=.25]{images/LogoUsach.pdf}}
\end{picture}

\sloppy
\thispagestyle{empty}
\vspace*{-1.6cm}

\begin{center}
	{\bf \mbox{\large UNIVERSIDAD DE SANTIAGO DE CHILE}}\\
	{\bf \mbox{FACULTAD DE INGENIER\'IA}}\\
	{\bf \mbox{DEPARTAMENTO DE INGENIER\'IA INFORM\'ATICA}}\\
\end{center}

	\vspace*{3cm}
	\par
	\vspace{1cm}
	\begin{center}
	\large
		Organización de Computadores\\Laboratorio \#1
	\end{center}
	\vspace{3cm}
	\begin{flushright}
		\begin{tabular}[t]{l l}
			Integrantes: & Nestor Mora \\
			             & Cristian Espinoza \\
			Profesor(a): & Nicolas Hidalgo \\
						 & Erika Rosas \\
			Ayudante: & Felipe Fuentes\\

		\end{tabular}
	\end{flushright}
	\begin{center}
		\vspace{3cm}
		Lunes, 6 de Octubre de 2014
	\end{center}



\newpage
\tableofcontents
\thispagestyle{empty}

\newpage
\renewcommand{\thepage}{\arabic{page}}
\setcounter{page}{1}
\section{Introducción}
El presente informe, detalla elaboración del Laboratorio 1 del ramo Organización de Computadores, el cual consiste en la producción de una aplicación que desarrolle el logaritmo natural de un número, pero resuelto a través del uso de la Serie de Taylor.

El objetivo principal es disminuir el tiempo de ejecución de la aplicación.

Los objetivos específicos son la elaboración de la aplicación que desarrolle la serie de Taylor para resolver el logaritmo natural de un número, detectar los hazards y la disminución en la cantidad de tareas realizadas en el proceso de calcular la serie.

Para disminuir el tiempo de ejecución de la aplicación, se ha procedido a considerar los hazards y a través del uso de pipeline, reordenar algunas instrucciones para que no exista la necesidad de esperar los resultados.
\newpage
\section{Marco Teórico}
\subsection{Serie de Taylor}
La Serie de Taylor es una serie funcional y surge de una ecuación en la cual se puede encontrar una solución aproximada a una función.

Ésta sirve para conseguir una aproximación del valor de una función en un punto.

\subsection{Pipeline}
Es una técnica de implementación en la cual múltiples instrucciones están traslapadas en la ejecución. Esto sirve para disminuir el tiempo de ejecución de un programa.
\newpage
\section{Desarrollo}
\subsection{Descripción del problema}
Según Wichterich boss supremo del universo, en su tesis de doctorado \cite{Wichterich2010} dijo...
\lipsum[1-3]

\subsection{Descripción del observatorio}
\lipsum[1-3]

\subsection{Descripción de las entidades y atributos}
\lipsum[1-3]

\newpage
\section{Modelos}
\subsection{MER: Modelo Entidad Relación}
\includegraphics[angle=90,scale=0.35]{images/MER.png}

\subsection{MR: Modelo Relacional}
\textbf{Entidades}

\begin{itemize}
\item Institución (Codigo\_institución, Nombre\_institución)
\item Carrera (Cod\_Carrera, Nombre\_Carrera)
\item Ramo (Cod\_Asig, Nombre\_Ramo, Semestre, Año)
\item Material (Cod\_Material, Visibilidad, Versión, Autor)
\end{itemize}

\textbf{Relaciones}
\begin{itemize}
\item Imparte (Codigo\_institución, Cod\_Carrera)
\item Contiene (Cod\_Carrera, Cod\_Asig)
\end{itemize}

\newpage
\section{Conclusión}
\lipsum[1-3]

\newpage
\addcontentsline{toc}{section}{Referencias}
\bibliographystyle{plain}
\bibliography{references}

\end{document}
